\RequirePackage[l2tabu,orthodox]{nag}
\documentclass[DIV=classic,11pt,numbers=noenddot,listof=totoc,bibliography=totoc,parskip]{scrartcl}
\usepackage{fixltx2e,multicol,graphicx,url,caption,csquotes,amsmath,paralist,ellipsis,subfig,array,microtype,flafter,siunitx,cleveref,booktabs,textcomp}
\usepackage[colorlinks,hypertexnames=false,plainpages=false]{hyperref}
\usepackage{fontspec}
\setsansfont[BoldFont={* Bold}]{Miso}
\setmainfont[BoldFont={* Bold},ItalicFont={* Italic},BoldItalicFont={* Bold Italic}]{Courier Prime}
\title{Efficiences of Different Market Structures}
\author{Linan Qiu}
\date{}
\begin{document}
\maketitle
\tableofcontents
\newpage
\section{Recap on Efficiency}
\subsection{Summary of Efficiency}
So far we’ve encountered a few types of efficiencies. Allocative efficiency is when we produce the right type and amount of goods in society. Productive efficiency is when we do so at the lowest costs possible. So here in this checkpoint we intend to go into them a bit further and formalize the definitions for them. So here we go.
\subsection{Pareto Efficiency?}
But before we go further, let’s dispense with a term first. Sometimes, your teachers use the term pareto efficiency. This is a term that usually results in the same outcome as allocative efficiency. Why? Because pareto efficiency simply refers to a situation where no improvement to welfare can be made without making somebody worse off. It is the optimum position. and that usually happens to be the position of allocative efficiency! But it does not simply mean allocative efficiency! Let’s bring up the PPC again. Now let’s say you’re producing at this point right now. What is pareto efficient? it is anything in this quadrant, because you can’t make anyone worse off! That’s good if the allocatively efficient point is within this quadrant along the line here. But What if the allocatively efficient point lies here? Then it won’t be pareto efficient, but it is allocatively efficient! This might be an oversimplified illustration, but it illustrates one of the limitations of pareto efficiency.
\newpage
\section{Allocative Efficiency}
\subsection{Review: Allocative Efficiency on a PPC}
So if allocative efficiency means to produce the right type and amount of goods in society, it should lie firstly on the PPC because you always want to produce as much as possible so that your consumers can enjoy more benefits! Then it is about getting the right combination of the two goods. 
\subsection{Review: Allocative Efficiency through Maximizing Surplus}
Allocative efficiency can also be express through a demand and supply diagram. At the equilibrium, the consumer surplus and the producer surplus is maximized. At anywhere else, the consumer and producer surplus will be lower. At a lower quantity, consumer and producer surplus will be lower like this. At a higher quantity, it will be lower as well. So this is the only point where surplus is maximized.
\subsection{Allocative Efficiency through P=MC}
Now let’s introduce a new concept. AE also happens when the price of your good equals to the marginal cost. Why is that so? The price of the good shows how much consumers value the good and how much benefit they derive from the good. The price shows the marginal benefit. The MC shows the marginal cost of providing the good. So if on the last unit of good, the benefits, the happiness or in economic terms the utility that the good provides is just nicely the cost of making it, then efficiency is maximized! Any amount less than that, P>MC, so society values the last unit more than what it takes to produce it. So society can benefit if an additional unit is produced. Any amount less than that, P<MC, so society values the last unit less than what it takes to produce it. So society benefits if less of it is produced. So eventually, it is only maximized when P=MC. You will then proceed to see how many market structures satisfy this condition for efficiency. 
\subsection{Allocative Efficiency through Cost Benefit Analysis}
In the next lesson, you will learn that there is a difference between social cost and social benefit. Then in this case, it is not just the Marginal Benefit that should equate the Marginal Cost. It is the Marginal Social benefit that should equate the Marginal Social Cost. You’ll get acquainted with this concept in the next topic.
\subsection{Conditions for Allocative Efficiency}
So to sum it up, we can express allocative efficiency using the PPC. We usually don’t use this too often. We can also express this through the supply and demand curve. We use this when analyzing a single perfectly competitive industry. AE also happens when P=MC. We usually use this when talking about market structures. Next is when MSB = MSC. We use that when we consider market failure and externalities. 
\newpage
\section{Productive Efficiency}
\subsection{Review: Productive Efficiency on a PPC}
Productive efficiency can be expressed as a point on the PPC itself, where the productive capacities of the country is being used to a maximum. Point A, B and C here are productively efficiency, while point D isn’t. That’s because point D is producing below the maximum potential of a firm or a country, depending on what you are using your PPC for. Some people are hence slacking in this situation.
\subsection{Representing Productive Efficiency using AC}
Productive efficiency can also be represented by the AC. When a good is being produced at the bottom point of the AC, productive efficiency occurs. That’s because the good is being produced at the lowest costs possible! At the same time, being at the bottom point of the SRAC doesn’t mean being at the lowest point of the LRAC, if you still remember this relationship between SRAC and LRACs. In that case to be productively efficient in the long run, the firm should always expand until it is at the MES. So here we have two requirements for being productively efficiency - being at the bottom of the SRAC and being at the bottom of the LRAC. 
\subsection{X-efficiency}
Many people confuse X-inefficiency with productive inefficiency. That’s because they occur in exceedingly similar situations. But here’s the definition of x-inefficiency. In economics, x-efficiency is the effectiveness with which a given set of inputs are used to produce outputs. If a firm is producing the maximum output it can, given the resources it employs, such as men and machinery, and the best technology available, it is said to be technical-efficient. x-inefficiency occurs when technical-efficiency is not achieved. The concept of x-efficiency was introduced by Harvey Leibenstein in his paper Allocative efficiency v. "x-efficiency" in American Economic Review 1966. So this might be different with productive efficiency, where all you are concerned with is producing at the lowest costs. 
\newpage
\section{Evaluating Market Structures}
\subsection{Structure Conduct Performance}
So now we go back to the structure conduct performance framework and we see how it all fits together for structure and conduct. We haven’t touched performance yet. Perfectly competitive firms have many firms, high freedom to entry and homogenous product. Monopolistic competition have many firms, high freedom to entry, and somewhat differentiated products. Oligopolists have few firms, high barriers to entry and homogenous or differentiated products. Monopoly is the only firm, very high barriers to entry and unique product. The special behavior of PC firms is that they are price takers. Monopolies are price setters because of their market power. Monopolistic competition engage in non price competition and are independent. Oligopolists are interdependent. This is very useful in answering questions that ask you to identify which market structure it is. Use both structure and performance and of course whatever data you have, for example the firm concentration ratios. Now we move on to performance in the next checkpoint. We evaluate them using 4 common criteria: efficiency, equity, consumer choice and innovation. 
\subsection{Efficiency}
So let’s look through the market structures one by one. Is the PC firm allocatively efficient? Well, yes! It always produces at P=MC, because it’s demand curve is its MR is its AR is its P, because an additional unit of good it sells always fetches that same price. So its P is always equals to MC. It does satisfy allocative efficiency! Society values the last unit of good it produces as much as it costs! So the PC firm is AE. Is the monopolistic competitive firm AE? Well almost. The P is only a little higher than MC, because it has market power. So you probably realize by now that since PC didn’t have any market power and Monopolistic Competitive firms have market power, market power causes allocative inefficiency to occur! That’s very true! What about oligopolists? Well they have a much more inelastic demand curve, similar to monopolies! Then in this case, their P is much greater than their MC. Society will benefit if more units of the good is produced, because for that last unit of good, they value it more than the cost. So they should produce more! But because they have market power, oligopolies and monopolies don’t do it. They under produce, causing an allocatively inefficiency outcome. What about productive efficiency? the PC firm might not be productively efficient in the short run because it might not produce at the bottom part of its AC, minimizing costs. The PC firm will always be productive in the long run because it always earns normal profits. It will always produce at the bottom of its AC in the long run. Same for monopolistic competition. In the short run no, but in the long run, always productively efficient because it always produces at the bottom of its AC. What about market structures that have market power like oligopolies and monopolies? Well, it has no incentive to be productively efficient! It doesn’t need to produce at the bottom of its AC because it can continue making profits by abusing its market power. So if it is productively efficient, it is only by chance, in this scenario Otherwise, it has no reason to be productively efficient! So it appears like, from only efficiency, PC is always the best firm! But is that so?
\subsection{Equity}
In terms of equity, PC firms also score the best as well! Since the firms are not earning supernormal profits, PC firms are equitable as they do not rip off consumers. But when market power increases to the oligopoly and monopoly spectrum of things, they actually do abuse their monopoly power, and charge consumers a price that is higher than the MC. They rip consumers off and allow themselves to profit, hence worsening inequity in society because usually those who own monopolies or oligopolies tend to be the rich already. 
\subsection{Consumer Choice}
The amount of choice the consumer has is important as well. We always want a diverse selection of goods to consume? Here’s where the PC firm sucks, because remember what we said! The PC firm can only produce homogenous goods. It has no reason to innovate, because whatever innovation it has will be copied immediately. Barriers to entry is zero, and other firms can just easily access information it has. So a lot of people actually prefer monopolistic competition, where there are minor variations in products. For example, I’m sure you won’t mind a little higher prices to enjoy 20 different kinds of food at many different restaurants instead of having the same food everywhere. Oligopolists may either have a whole range of goods, or only one. It really depends on the firm itself. Monopolies, by definition is the only supplier. However, the monopoly can provide a range of products. Most of the soft drinks you consume, sprite and cola, are both marketed by Coca Cola. In this case, is there still consumer choice? Some say yes because you get a lot of goods. Some say no, because well at the end of the day you’re still buying from the same firm. That’s really up to you to decide. 
\subsection{Innovation}
We alway want firms to come up with innovations, to advance and be dynamic. The PC firm sucks at that, because again, no one has any incentive to innovate. The monopolistic competitive firm only innovates a little, for example cooking dishes differently or providing different types of services. The oligopoly and the monopoly are the only ones who can innovate significantly because of the capital required. But sometimes they don’t because they can already earn enough! So it really depends on how we can make monopolies continue innovating and introduce the threat of new entrants or copy cats to make them keep their barriers to entry high by constantly innovating.
\subsection{Contestable Markets}
This brings us to an interesting concept called Contestable Market. Monopolies might actually be made competitive! So here’s how. A market is perfectly contestable when the costs of entry and exit by potential rivals are zero, and when such entry can be made very rapidly. In such cases, the moment it becomes possible to earn supernormal profits, new firms will enter, thus driving profits down to a normal level. The sheer threat of this happening, so the theory goes, will ensure that the firm already in the market will keep its prices down, so that it just makes normal profits, and produce as efficiently as possible, taking advantage of any economies of scale and any new technology. If it did not do this, rivals would enter, and potential competition would become actual competition. Now does this happen in real life? Why in such cases are the markets not actually perfectly competitive? Why do they remain monopolies? The most likely reason has to do with economies of scale and the size of the market. To operate on a minimum efficient scale, the firm may have to be so large relative to the market that there is only room for one such firm in the industry. If a new firm does come into the market, then one or other of the two firms will not survive the competition. Barriers to entry costs may actually be significant. A study has been done that shows that 5\% of the sunk cost can be enough to deter entrants. In this case, how can contestable markets happen? 
\subsection{Natural Monopolies}
We have mentioned Natural Monopolies before. Here we would just like to point out that many times, Natural Monopolies are actually legitimate reasons for monopolies to exist. Two bus companies might find it unprofitable to serve the same routes, each running with perhaps only half-full buses, whereas one company with a monopoly of the routes could make a profit. Electricity transmission via a national grid is another example of a natural monopoly. Even if a market could support more than one firm, a new entrant is unlikely to be able to start up on a very large scale.
\end{document}