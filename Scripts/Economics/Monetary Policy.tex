\RequirePackage[l2tabu,orthodox]{nag}
\documentclass[DIV=classic,11pt,numbers=noenddot,listof=totoc,bibliography=totoc,parskip]{scrartcl}
\usepackage{fixltx2e,multicol,graphicx,url,caption,csquotes,amsmath,paralist,ellipsis,subfig,array,microtype,flafter,siunitx,cleveref,booktabs,textcomp}
\usepackage[hypertexnames=false,plainpages=false]{hyperref}
\usepackage{fontspec}
\setsansfont[BoldFont={* Bold}]{Miso}
\setmainfont[BoldFont={* Bold},ItalicFont={* Italic},BoldItalicFont={* Bold Italic}]{Courier Prime}
\title{Monetary Policy}
\author{Linan Qiu}
\date{}
\begin{document}
\maketitle
\tableofcontents
\newpage
\newpage
\section{    Interest Rates}
\subsection{A Quick Analogy}
Welcome to the topic of monetary policy. Before we go into monetary policy, which means the manipulation of interest rates to induce certain effects in the economy, let’s understand interest rates first. Let’s say that you lend me \$5000 this year. 5 years later, will you want only \$5000 back? Or will you want more? You will probably want more. Why? Because you have parted with \$5000 for 5 years. What could you have done in these 5 years? You could have taken a trip to somewhere. Let’s say Paris. I heard Paris is good at this time of the year. Or you could have invested the \$5000. Over 5 years, it could have earned you a return of 20\%. So at the end of 5 years, supposing that you kept the cash instead of giving it to me, then you would have \$6000. So, by lending me the cash, instead of earning money, you forgo the additional \$1000. That’s an opportunity cost incurred by lending me the money. So you’d want me to return you \$6000 instead of \$5000 after 5 years. So you impose a 20\% interest rate over 5 years. So when you lend me \$5000 now, I return you \$6000 later. That additional \$1000 is compensation for you letting go of the money for 5 years. If I call those additional 20\% on top of the principal amount, that is the original \$5000, then the interest rate can seen as the compensation for you parting with the \$5000 for 5 years. 
\subsection{A More Interesting Definition of Interest Rate}
Now let’s think from the borrower’s perspective. The interest rate can be thought of as the price of borrowing money. Why? If I borrow \$100, and the interest rate is 2\% per annum, that means next year, I will have to return \$102. The extra \$2, or the additional amount that can be attributed to the 2\% interest rate, is the price of borrowing \$100. Let’s try another example. If I borrow \$5000, and the interest rate is 10\% per annum, then after one year, I will owe the lender \$5500. The \$500, or the interest, caused by the interest rate of 10\%, is the price of me borrowing that \$5000. 
\subsection{Demand of Money}
Now money has no value by itself. It is just a piece of paper. Really? Why do people say, “I want to have more money!” then? Well, they don’t actually want money. They want the things that money can buy. They want cars, food, and cookies. And money is precious only for the the things it can buy. So it is natural that the demand of money depends on the real economy. Well, turns out it does. The demand of money depends on 2 things - transaction and assets. First, transaction cost. In order to get money, you go to the ATM. Let’s say there’s only one ATM in the country you live in. Then, it is very inconvenient for you to go to the ATM. The cost of one transaction of withdrawing money is very high. So when you go to the ATM, you’d probably want to draw a lot of money at once. If, however, there are many ATMs around, or it is simply very convenient for you to collect money, then the transaction cost of money is very low. Then, you won’t demand a huge amount of money at once. In fact, you’d probably keep very little money in our wallets (as we do in most modern countries) and have a low demand for money.  So if transaction costs are high, there will be high money demand. If transaction cost is low, there will be low money demand. Now let’s assume that the transaction cost doesn’t really change (because in reality, it does not really change through time!) Then money demand depends on the second thing - assets. Let’s see. When you hold on to money in your pocket, is your money in the bank? No! It is in your pocket as cold hard cash. You are not saving it in the bank, nor are you investing it anywhere. Hence, you are losing the returns you could have got by investing it or by saving it - the interest rate that you can get on these cash. Hence, if interest rates increase, would you prefer holding on to the money and foregoing all the returns you can get, or will you want to save more money (or invest it somewhere else?) You will probably want to not keep the money in your wallet and put it in a bank. Hence, money demand is low when the interest rates are high. Similarly, when interest rates are low, the cost of holding on to money is low, hence money demand will be high. If we assume that transaction costs are constant, then the asset motive will cause the money demand curve to slope downwards. Money demand will then be low when interest rates are high, and money demand will be high when interest rates are low. Hence the shape of this money demand. When high interest rate, low money demand, low interest rate, high money demand. 
\subsection{Supply of Money}
The supply of money is determined by the central bank in an economy. Without counterfeiters (which do not make up a significant portion of the money supply in most economies), the central bank in an economy, for example in the States that’s the Federal Reserve, prints all the money in an economy. There are many tiers of calculation of money: M1, M2, M3 for example. However, all you have to know is that money supply is determined by the central bank. So if the money supply is determined by the central bank, then it is vertical line here, simply because the central bank sets the money supply in an economy! 
\subsection{Short Term Interest Rate}
The intersection of the money demand and the money supply gives the short run interest rate. Why? Because any interest rate above this, money supply will be higher than money demand. Then there will be more people willing to supply money then demanding money. Now think about this. If I’m a bank, and I want people to hold on to more money, I must lower the interest rate! If not, people will find it too expensive to hold on to money because interest rate is the price of holding on to money. So, I lower interest rate, and quantity of money demanded increases, and we move towards the equilibrium. Now let’s think about the other scenario where money demand is higher than money supply. If money demand is higher than money supply, then there will be lost of people queuing outside the bank to withdraw money. How do I convince these people that it’s not worth their time? By increasing the interest rate! So interest rate increases, and less people queue outside the bank, and eventually when no one queues outside the bank, we reach the new equilibrium. Hence the intersection of money demand and money supply gives us the short term interest rate. Now if the central bank moves its money supply by changing the amount of money in circulation, interest rate changes! For the A Level syllabus, you are not required to know how the money changes, but we know you’re interested. So in the last lecture of this topic, we included a special checkpoint about manipulation money supply. In that checkpoing, we explain the mechanisms through which the central bank change the money supply. Do watch that if you’re interested in the money supply. But for now, let’s assume that the central bank can manipulate the money supply. Then, changes in money supply causes changes in interest rates. Now why only short term interest rate? To understand this distinction, we need to understand long term interest rate and the difference between short term and long term interest rates. 
\subsection{Long Term Interest Rate}
The long term interest rate is the underlying interest rate in an economy. Think about it this way. In the short run, interest rates fluctuate according to the demand and supply of money. But in the long run, it will always tend to a long run interest rate. This interest rate is determined by economic fundamentals. What do I mean by economic fundamentals? One is the risk of investment. Let’s say you’re a businessman in Myanmar. Is it safe to do business in Myanmar? Well a joke that The Economist made was that the safety of one’s investment used to be expressed in the number of stars on your business partner’s shoulder -- a reference to the junta regime in Myanmar before it liberalized. So before liberalization (and perhaps even after), it is risky to do business in Myanmar. In that case, if you ask for money from a bank to fund your investment, the bank will charge you higher interest rates! Why? Because it is highly likely that you will not get the money back! Hence the bank charges you high interest rates to justify that risk of lending you money. Now imagine that all the investments in one country is highly risky. Then the interest rate in that country will be persistently high. Similarly, if it is safe to do business in a country, then borrowers will only ask for a moderate interest rate because they know that their money is safe. More people will be willing to lend money to the investor, and that drives interest rates down! So risk is one factor in determining long term interest rate. Long term interest rates also take into account return on investment of capital. Let’s say that you have \$1000. You can choose to either invest in a factory, or you can save it in a bank. Factory gives you 10\% returns, and bank only gives you 5\%. In that case, does the bank have an incentive to increase its interest rate in order to make you save? Of course it does! On the other hand, if the bank offers you a higher interest rate, say 15\%, then many people will offer its money to the bank. The bank can afford to give a lower interest rate. So interest rate follows return on investment as well. So the long term interest rate is driven by economic fundamentals like these. 
\subsection{Difference between Short Term and Long Term Interest Rate}
We have just defined short term and long term interest rate. While money demand and supply affects the short term interest rates, the long term interest rates are driven by economic fundamentals. What this means is that the short term interest rates may fluctuate, but they will always tend towards the long term interest rate in the long run. So like in this graph, in the short run, money demand and supply determines interest rates, and may produce a fluctuating interest rate like this. However, it always tends towards and is driven fundamentally by the long term interest rate which is determined by economic fundamentals. So in the short run, let’s say that money demand soared due to an increase in GDP. In other words, you earn more money, and hence you want to hold on to more money. Then, interest rates might increase. However, in the long run, it will go back to the lower interest rates because the underlying factors -- risk and return to capital -- have not changed. This has important implications for the interpretation of monetary policy which you will learn in a few lectures.
\newpage
\section{Money Neutrality}
\subsection{Classical Interpretation of Money}
Classical economics is the traditional old school economics. Starting with the great grandfather of economics Adam Smith, classical economists claimed that free markets regulate themselves, when free of any intervention. Adam Smith referred to a so-called invisible hand, which will move markets towards their natural equilibrium, without requiring any outside intervention. One of the things they believe in is that prices are flexible. This belief has great implications on their interpretation of money. When money supply is increased, they believe that price levels will increase. In other words, there will be inflation. This happens because there is more money in the economy, but the real goods -- the amount of cookies that money can buy in the economy -- is still the same. Say originally there are ten cookies, and there are \$100 floating in the economy. Naturally, each cookie costs \$10. But now if the government increases the money supply and there are \$200 floating in the economy, then each cookie costs \$20! So there will be an increase in price level, in other words inflation, if the government increases the money supply. At the same time, remember our graph for money demand and supply? When the money supply increases, there will be a decrease in interest rate. Yes the money supply does shift to the right, but because of higher price levels or inflation, people demand more money! If cookies now cost \$20 each, people will require more cash at hand to buy them. Money demand increases as well, and interest rate stays the same! So eventually, the short run interest rate never changes due to money supply movements! This is all because classical economists believe that price levels are flexible --- that when money supply increases, price levels necessarily increase as well. Short run interest rates cannot be subject to manipulation by the central bank, and hence always follows the long term interest rates. In other words, those two are one and the same to classical economists.
\subsection{Keynesian Interpretation of Money}
Keynesians however do not believe in the same stuff. Keynesians economics are the group of macroeconomic schools of thought based on the ideas of 20th-century economist John Maynard Keynes. Besides having a cooler name, they believe in many things that are different from Classical economics. One of them is price flexibility. Keynesians believe that prices are sticky in the short run. Why? Well ask your economics teacher that if the economy slows down by 5\% in a year, will he get a wage cut? He probably will say no! He is bound by a contract to the school that lasts 3 years at least. The same applies to most workers in modern day economies. Similarly, do businesses change their prices all the time? Every time they change a price, they have to change their catalogues, displays and restructure their entire marketing strategy. We call these menu costs. These prevent firms from changing costs in the short run. Take for example the bus company. They don’t change their prices all the time! They only do it every once or twice a year, not whenever fuel prices change (and that is almost on a daily basis). This is price stickiness. If prices are sticky, then even though money supply increases (or decreases), in the short run, there will be no changes to the price level. Let’s say that the money supply increases. The price of cookies won’t increase because of price stickiness. Then, in the domestic money market, the supply of money increases, but the demand of money doesn’t. Short run interest rate drops! Similarly, if money supply decreases, then interest rate increases! However, do Keynesians believe that prices are always sticky? Actually no! They do believe that in the long run, businesses and employees adjust their prices. Hence, in the long run, prices will adjust to reflect the changes in money supply. In other words, in the long run, if there is a money supply increase, then inflation will happen and cause interest rates to return back to normal. The same happens for a decrease in money supply. In the long run, Keynesians believe in the same things as Classical economists. However, this highlights the difference between short run and long run interest rates. Classical economists believe that short run and long run interest rates are essentially the same things, and there is no way to manipulate short run interest rates. Keynesian economists however believe that short run interest rates can be manipulated, but in the long run they will still tend towards long run interest rates.
\subsection{Which Interpretation?}
Which interpretation are we using in openlectures? Well, it definitely makes more sense that the prices of goods and services don’t change overnight and all the time! There is a certain amount of price stickiness in the economy. Hence, in openlectures as with most economists, we do believe that price stickiness happens. In this case, changes in money supply in the short run do not affect price level. Let’s review this again. Let’s say money supply increases. Then, interest rate drops in the short run. However, in the long run, because of the money supply increase, there is inflation! There is inflation in the long run because prices are no longer sticky in the long run. After the inflation, money demand increases and eventually erode the interest rate decrease, causing interest rates to drop to the original level and to match the long run interest rate. Hence, short run interest rates drop when money supply increases, but eventually still gravitate towards long run interest rates. Similarly, when money supply decreases, interest rate increases in the short run. However, in the long run when prices are more flexible, money demand decreases due to decreasing prices. When money demand decreases, it lowers interest rate until the short run interest rates matches the long run interest rate. Hence in the long run, only the long run interest rate is reflected in the economy.
\newpage
\section{Monetary Policy}
\subsection{Definition of Monetary Policy}
Let’s define monetary policy proper now. Monetary policy is the deliberate manipulation of interest rates in an economy to influence economic growth and other macroeconomic goals. How does the government manipulate interest rates? I’m sure you’re an expert by now. Remember the movement of the money supply? Good. So let’s say that the interest rate right now is 3\%. If we want to increase the interest rate to 5\%, what should we do? We should decrease the money supply to just the right amount until interest rate hits 5\%. Similarly, if we want to decrease the interest rate to 1\%, we adjust the money supply as well. This is why sometimes you will hear your economics teacher saying, “You can only adjust either the interest rate or the money supply, but not both!” This is because when you desire to change the interest rate, you are dictated by the interest rate and the money supply HAS to change by a given amount. You can’t have freedom on both interest rate and money supply! Choose one, the other will have to follow. Most of the time, governments target the interest rate. The changes in interest rate has many implications on the 4 macroeconomic goals. Let’s look at each one in turn. First we will look at economic growth, and to do that, we will look at the 4 constituent portions of economic growth.
\subsection{Effects on Consumption}
You usually buy two kinds of items: durables and non durables. Are they really that different? Well, cars and mars bars are really different aren’t they? That’s due to the percentage of your income they take up. Cars take up a rather significant portion of your income and they last for a long time. We call them durables. Mars bars don’t really take up that much of your income and they can’t last for very long. This is especially so in Singapore -- the weather simply melts a mars bar in ten minutes if you leave it out in the open. For durables and other big ticket items, you often find that you have to spend a significant amount of money on it -- so you either spend a lot of your savings or you borrow money. Let’s say you borrow money first. Then, if interest rate increases, the cost of borrowing increases. You have to return more money to the lender next time. In that case, when interest rate increases, your consumption of durables decreases. Now an interesting question you may ask is: what if I don’t borrow the money and I simply spend the cash that I have already or my savings? Well, the same reasoning still applies. Let’s say a car costs \$100,000. Let’s say that you have \$100,000 cash on your hand right now. What else can you do besides buying the car? You can save it in a bank and earn interest on the money. Hence if you spend money on the car, you forego the interest that you could have earned saving this money in a bank. What does that mean? This means that the interest rate is also your opportunity cost of present consumption. By consuming the \$100,000 now, you are giving up the interest you could have earned. When interest rate increases, say from 5\% to 10\%, then you are giving up a larger interest by consuming instead of saving. You are giving up \$10,000 instead of \$5,000. In that case, you will be less inclined to consume due to the higher opportunity costs. You may save instead. Hence, when interest rate increases, similar to the case when you borrow, your consumption of durables still decreases. What about mars bars? Well, I don’t think interest rates have much to do with your consumption of mars bars. You rarely borrow to buy mars bars unless you’re still in primary school. Neither do you forgo too much of an interest either. What is 1\% of \$2? Hence, interest rates only significantly influence the consumption of durables. When interest rates increase, consumption of durables decrease. When interest rates decrease, consumption of durables increase. Simple as that.
\subsection{Effects on Investment}
Let’s say you’re an investor thinking of investing in...cookies. The rate of return of your investment is 5\%. Now suppose you have to borrow money to finance this investment. If the interest rate is 4\%, that means that the rate of return of your investment is higher than the interest rate. This means you get to pocket some money and your investment will be profitable! Now let’s see what happens when the interest rate is 6\%. In that case, even though you earn 5\% on your investment, the interest rate is higher than your rate of return hence the investment actually loses you money instead of earning you money. Then, you’d find that the project is unprofitable. Now suppose you don’t have to borrow money but you have a stock of cash -- usually companies call this retained profit because it is leftover profit from previous years. You don’t have to pay interest on this money, but again you have two choices -- you can either invest the money in cookies, or you can save it in a bank. Save it in a bank, you earn interest rate! Hence, if the interest rate is 4\%, you’d rather take your money and invest in cookies. However, if the interest rate is 6\%, you’d rather save your money and not invest in cookies. Hence, you can see that if I increase the interest rate, there will be less investments because you’d have to find even more profitable projects to invest in. Let’s say there are a 100 projects, and out of these only 50 can earn you at least 5\% rate of return. When I increase the interest rate to 6\%, these 5\% rate of return projects are no longer profitable! So you have to move to projects with at least 6\% rate of return, and that’s only 30 projects. Hence, when I increase interest rate, the number of profitable projects decreases! This also decreases investment because there will be less projects in which investors can throw their money into. There is a formal theory to describe this. It is called the Marginal Efficiency of Investment. On the Y axis, we plot rate of return. On the X axis, we plot amount of investment. Let’s say that at 10\% rate of return, there’s only 1 profitable project. At 9\%, there are 2. At 8\%, there are 3. So on so forth until at 1\%, there are 10 projects. Now since the rate of return always has to be more than or equal to the interest rate, because you only invest in profitable projects, then let’s say the interest rate is at 5\%. You’d only invest in projects that earn you 5\% or more. Hence, you’d only invest in 6 projects. Let’s say the interest rate is at 3\%. Then, you’d invest in 8 projects. Hence, the Y axis can also show the interest rate. This graph hence shows the relationship between interest rates and amount of projects, or formally called the amount of investment. Hence, when interest rates increase, amount of investment drops. When interest rates decrease, amount of investment increases.
\subsection{Effects on Government Spending}
Let’s say you want to try and borrow money from me. Say...a \$1000 so that you can go on a really really posh date with your girlfriend or boyfriend or both. If I charge you a very low interest rate, you’d go, “Oh hell yeah! I want to borrow money from you!” But if I charge you an insane interest rate of 25\%, you will say, “Nvm. Let’s go eat at the food court instead.” The same applies to the government! When the government runs a deficit -- now if you don’t know what that term means, I suggest you go through fiscal policy again. So when the government runs a deficit, it has to find a way to finance such a deficit. One way is through borrowing. So if interest rates are high, the costs of borrowing are high, so governments will find it very very costly to fund a deficit! This happens if people think that the government is close to bankruptcy and will not be able to make good on its loans. An example is the countries involved in the European fiscal crisis. They all have very high interest rates that made it extremely difficult for them to borrow on the open market. In this case, the government may hesitate before spending a lot of money. Hence, higher interest rates may lead to lower government spending. The reverse happens as well.
\subsection{Effects on External Balance}
Now interest rates have a peculiar effect on external balances. The reasoning is long, but very logical so follow me. Let’s say there are two countries -- US and UK. US uses dollars, and UK uses pounds. Let’s say that the interest rate in US is 5\%, and the interest rate in UK is 3\%. If there's a British chap, will he prefer to save your money in UK or in US? He'd prefer to do that in US because it gives him higher interest rates than those he gets domestically. So the British chap will move his money over to the States. What happens then? American banks don’t accept pounds! So he'll have to change his pounds into USD. He will sell his pounds in the forex market and exchange that for USD. When that happens, the demand for USD increases, and the e/r of USD against the Sterling Pound rises. This has implications for the American economy. If the USD appreciates against the Sterling Pound, it will take more pounds to buy one USD. The price of American exports in Sterling Pounds increases, and supposing that the exports are price elastic, then quantity of exports will decrease more than proportionately. This causes a drop in value of exports. Also, when the USD appreciates against the Sterling pound, when Americans import goods from UK, the goods are cheaper because one USD can now buy more Sterling pounds. The price of imports in USD decreases, and supposing that imports are price elastic, then the quantity of imports will increase more than proportionately. This causes the value of imports to increase. If the value of your exports decrease and the value of your imports increase, this worsens your net exports (X-M.) And this all happens when America's interest rate is RELATIVELY higher than those of UK. In other words, when your interest rates are relatively higher than those of other countries, your net exports will decrease. Let's think about the other scenario. Let's say the interest rates in US is 2\% this time, as compared to UK's 3\%. Then in that case, the savers in US will prefer to save in UK. Then, he will want to transfer his funds to UK. We call these funds that are being transferred Short Term Capital. There is an informal term for them -- hot money. They are called such because they switch very quickly whenever there are interest rate fluctuations. If he transfers his funds from US to UK, he will sell USD to buy pounds. When he sells USD, he increases the supply of USD on the international forex market. Note that this forex market is different from the domestic money market we are talking about all along. One involves exchange rates as we are talking about now and occurs internationally. The other only involves one currency -- that of your own, and it occurs only domestically. It is also only concerned about interest rates. Now going back to the forex market, because he sells USD, the supply of USD increases, and this causes the USD to depreciate against the Sterling pound. What happens then to the US economy? The price of US exports in pounds decrease because one USD can now buy fewer pounds. In other words, one pound can buy more USD. Then, if the demand for exports is price elastic, the quantity demanded of exports will increase more than proportionately. Value of exports will then increase. Similarly, the price of imports in the States from UK will increase in USD, simply because one USD can buy less pounds now. If that happens and imports are price elastic, quantity of imports decrease more than proportionately. Value of imports will decrease. Then, net exports for USD will increase, and this all occurs when US has a relatively lower interest rate. Note that we are only concerned about RELATIVE interest rates here, not absolute. In other words, it does not matter whether US has a high interest rate or not. It only matters if US has a higher interest rate or lower interest rate THAN another country -- in this case UK. The effects happen due to the flow of short term capital or hot money.
\subsection{Effects on Economic Growth}
Since AD = C + I + G + (X-M), and we have analyzed what happens to each of these components in turn, let's combine their effects. Suppose interest rates decreases because money supply is shifted to the right. Consumption increases because durables become cheaper, both when you borrow money and when you use savings. Investment increases because there are more projects now that are profitable. Government spending increases because it is cheaper to borrow money in the market. If the lowered interest rate makes your interest rate lower relative to those of other countries (and this is very important because only relative interest rates matter when we talk about effects on net exports), then short term capital flows out of your country. Your currency gets sold on the forex market, supply increases, exchange rate drops relative to that of your trading partners, your exports become cheaper in foreign currency and your imports more expensive in domestic currency, the value of your exports increase and the value of your imports decrease. Hence, your net exports increase. We can see that they all point in one direction! When interest rates decreases, your AD increases because each of the components of your AD increases. What happens on your AD AS graph then? Well, when your AD increases, your national income increases as well, but by a more than proportionate amount. Why? Because of the multiplier effect! Now if you forgot what that is, this simply means that the spending of one person becomes the income of another. When that happens, when one person's income increases by a certain amount, the overall spending increases by much more than that amount. For the specifics, refer to the income determination lectures! The reverse happens when your interest rate increases. Try figuring that out for yourself! It shouldn't be hard at all. Now let’s introduce two terms. Instead of saying, “Lowering the interest rate so that the economy will grow,” we can say expansionary monetary policy. The word expansionary is used because the policy is designed to expand the current production in the economy. Similarly, contractionary monetary policy means increasing interest rates to lower production. These are merely terminology! Do not be afraid of them. Is that all? That’s actually not all. Note that we have only been talking about actual economic growth. What about potential economic growth -- that is the growth of the maximum capacity in the economy? The effects that monetary policy has on the productive capacity of the economy is minimal. It may spur further investment, but there’s no way of making sure that this investment goes towards increasing the productivity of the economy. Same goes for government spending. Consumption and net exports do little to improve productivity. Hence, monetary policy has little effects on potential growth.
\subsection{Effects on Unemployment}
Unemployment is pretty simple to explain. Unemployment basically follows national income. Hence, if interest rate goes up, AD goes down, national income goes down even more due to the multiplier effect. Then, unemployment will go up because less people will be required for production in the economy. Similarly, if interest rate goes down, AD goes up, national income goes up even more due to the multiplier effect. Then, unemployment decreases because more people will be required for production in the economy. However, is it that simple? This is a rather superficial analysis. We can dig deeper in two areas: one being the type of unemployment that monetary policy affects, and the other being the position of the AD on the AS. The only type of unemployment that monetary policy dictates is cyclical unemployment. It does nothing to facilitate the flow of information and help people transit between jobs. Hence, it does not do anything to frictional unemployment. Neither does it help people gain new skills. It does not do anything to structural unemployment as well. However, it does affect demand driven unemployment. It helps push the economy towards full employment. Now you do remember that full employment means zero cyclical unemployment, but not zero frictional and structural unemployment right? Good. So it only affects cyclical unemployment. Second, the effects on employment depends on the position of the AD on the AS. Let’s say that the AD is in the Keynesian region, hence there are lots of unemployed people in the market and there’s a lot of spare capacity in the economy. Then, when you decrease the interest rate and exercise expansionary monetary policy, you can decrease cyclical unemployment because there is cyclical unemployment in the first place. However, if you’re already producing at the full capacity of your economy and everyone is already employed, how can you still employ more people? Hence, the effects of an expansionary monetary policy, if the central bank is stupid enough to do that in a time of full production, will be purely inflationary.
\subsection{Effects on Inflation}
The effects of monetary policy depend large on which section of the AS you are in, just like the effects on unemployment. However, the section of the AS your AD resides in also reflects your view of the economy -- the Keynesian region is so called because it reflects Keynesian views. In the Keynesian model, remember we said that price levels are constant in the short run? This is reflected in this AS. No matter how the AD moves, the price level is always constant. Hence, even if you increase or decrease the interest rate, and the AD shifts accordingly and so does the national income, the price level is always constant. The classical region of the AS, however, reflects the long run situation in the classical school of thought. In the long run, the economy tends towards full employment, and this is the scenario that the classical school of economics describes. In it, if I increase the interest rate, the AD shifts down, the national income shifts down even more. This has cooling effects on the economy, and hence there is a reduction in price level. This is often used when the economy is overheated (and near full capacity). If I decrease the interest rate, eventually national income increases and I will have inflationary effects. If I’m already at complete full employment, then the effects are purely inflationary and there will be no increase in national income. Hence, it really depends on which region of the AS your AD is in. Generally, during a recession, your AD tends to be in the Keynesian region because there is lots of slack in your economy. During an economic boom, your AD tends to be in the Classical region because you are experiencing near full employment. There will be very low cyclical unemployment. The analysis for inflation then differs.
\subsection{Effects on External Balance}
We have already explained how monetary policy can influence your current account --- that is your net exports. Let’s formalize this in more technical terms, using the terms domestic for yourself and foreign for your trading partners. Remember what we are concerned with here is relative interest rates, not absolute interest rates. It is how high or low your interest rates are relative to someone else, not the absolute level of interest rates that you have in isolation. So let’s say that domestic interest rates are lower than those of foreign interest rates. What happens? Your domestic people will transfer their savings to foreign banks. In so doing, they will sell your domestic currency on the forex market. This causes the supply of domestic currency on the forex market, and remember that this market is different from the domestic money market. This causes the supply of domestic currency to increase, causing your currency to depreciate against foreign currencies. The price of exports in foreign currency becomes cheaper, and the price of imports in domestic currency becomes more expensive. If both demand are price elastic, then the quantity of exports increases more than proportionately and the quantity of imports decrease more than proportioantely. This causes the value of exports to increase and the value of imports to decrease, eventually causing your net exports to increase. This improves your current account. Since the current account is a constituent portion of the overall balance of payments, then balance of payments tends to improve (But this still depends on the other accounts). A similar reasoning can be employed for the case where domestic interest rate is higher than foreign interest rates. If this happens, there will be an inflow of hot money as foreign investors will want to save their money in your country. They sell foreign currency on the forex market and buy your own currency. When this happens, the demand for domestic currency increases, and domestic currency appreciates against the foreign currency. Price of exports in foreign currency increases, and the price of imports in domestic currency decreases. Quantity of exports decrease more than proportioantely and the quantity of imports increase more than proportionately. This causes the value of exports to decrease and the value of imports to increase. This worsens your net exports, and causes your trade balance to worsen. This contributes to a worsening balance of payment. Now an expansionary monetary policy may make your interest rate lower than those of other countries, but this is not absolutely true because it still depends on the interest rates of your trading partners. Similarly, when you apply contractionary monetary policy, you may cause your interest rates to be higher than those of your trading partners, but again this depends on your RELATIVE interest rate, not the absolute one.
\newpage
\section{Evaluation}
\subsection{Sentiments}
So it seems like I can simply manipulate my entire economy just by changing the interest rate. Isn’t that convenient and magical? Well, sort of. But there are problems. Now let’s evaluate if monetary policy is really as magical as it seems. The first issue we discuss is that of sentiments. Sentiments can affect the workings of monetary policy. Let’s say that you’re trying to enact expansionary monetary policy. You lower your interest rates, and hope that your C, I, and X-M increases. Well it may. But by how much? Let’s say that you already have a recession that is on going. People feel like crap already. Who’s going to buy a car in this climate? So even if you lower the interest rates say from 2\% to 1\%, and that is already a huge huge drop in the world of economics, people are not going to buy a car just because of that! C simply may not increase. Same for I. Even though there are more profitable projects now, but because of the investment climate and the fact that investors are usually risk averse, they will not invest unless there is a huge and obvious profit margin there. We can express this as a rather steep MEI. Even though interest rate changes, there isn’t much change to the level of investment. Hence, sentiments play a part in the effectiveness of monetary policies.
\subsection{Liquidity Trap}
The second evaluation we can use is that of the liquidty trap. Let’s bring up your money demand and supply graph again. Notice this flat portion? The idea is that even though people demand money for 2 reasons -- transaction and asset, but there is a fixed amount they will need no matter what. That is the transaction amount. Hence, it is the asset one that varies. This leads to a plateau in the region on the right. What this means is that even if you increase money supply all the way to the right, people will still need money for transaction (even though they may not desire it as an asset). This sets a lower maximum for the interest rate. Interest rates cannot possibly go any lower than this. What then happens is that if you try to do expansionary monetary policy, the interest rate won’t budget at all! If the interest rate doesn’t budget, you are handicapped! There is nothing you can do to affect the AD if your interest rate doesn’t move. We call this region the liquidity trap, because it is a trap that renders interest rate policy useless. This limits the effectiveness of expansionary monetary policy.
\subsection{Specificity}
Let’s say that you have a recession caused by structural unemployment. In other words, originally, your economy was manufacturing heavy. There were many workers and they all worked in factories. Now, you require more people in the services sector and hence these people who only had skills suitable for factories are unemployed. This gives you structural unemployment. Is monetary policy a good solution? Not at all! Monetary policy only cures cyclical unemployment. It does not do anything to impart new skills to people or increase the type of vocational training available in the economy! It simply spurs people on to spend more. It tackles cyclical unemployment not structural unemployment. In this case, something else should be used instead of monetary policy although both are reflected as unemployment rates increasing. 
\subsection{Sticky Price}
The stickiness of prices also affect the effectiveness of monetary policy. If prices are completely flexible like the classical economists believe so, then when money supply increases, interest rate does decrease in this graph. However, the increase in money supply causes price levels to increase because there will be more money chasing the same amount of goods. When price levels increase, money demand increases because you’d need more money to conduct transactions in higher amounts. If cookies originally cost \$10 and now it costs \$20, you’d need more money! Money demand increases, and effectively erodes the original increase in interest rates. Hence, short term interest rates can never be manipulated in the classical model. If you subscribe to this theory, then monetary policy is effectively useless to you. However, I’d advise you not to do this because as we have discussed earlier, complete price flexibility is more of a theoretical tool than a description of reality. It does not happen in reality. In fact, price stickiness is more of a reality. In this case, we go to the Keynesian model that assumes price stickiness in the short run. If prices are sticky in the short run, then when money supply increases or decreases, and we suppose that it increases this time, interest rates decrease. The increase in money supply does not trigger increase in price levels in the short run because prices are sticky in the short run! Hence, this lower level of interest rate is present in the short run. However, in the long run, when prices are less sticky, money demand increases and this erodes the original drop in interest rate. Hence, while monetary policy may be effective in controlling short term interest rate, it cannot affect long term interest rate and will only contribute to inflation. This is one argument you can use against monetary policy.
\newpage
\section{Additional Topics in Monetary Policy}
\subsection{Singapore and Monetary Policy}
This lesson contains several additional topics on Monetary Policy that are not strictly in the A level syllabus, but are very interesting and I strongly recommend you watch them. First, does Singapore employ monetary policy? No it doesn’t! It only employs exchange rate policy. We will explain why Singapore uses exchange rate policy when we introduce exchange rate policy to you later. But first, why does it not exercise monetary policy? Exercising monetary policy and manipulating your domestic interest rate risks making your domestic interest rate different from those of other countries. Now you do remember what happens then right? There will be the inflow and outflow of hot money that severely impacts your exchange rates. There we find problems. First, Singapore is heavily dependent on imports and exports. A fluctuating exchange rate caused by the inflow and outflow of hot money is damaging for both domestic consumers and exporting businesses. Hence, Singapore will prefer to just let our domestic interest rate float to match the equilibrium interest rate internationally so that there will be no inflow and outflow of hot money. Second, the huge inflow and outflow of hot money will be damaging on its own to an economy. These money cannot be depended on for investment -- it withdraws way too quickly. A small economy like Singapore cannot sustain sudden withdrawals of capital as well. Hence, it is better for Singapore to have the same interest rate as the equilibrium international one. This is why we do not do monetary policy. 
\subsection{Impossible Trinity}
I have seen some students who write in their essays, “The country should implement monetary policy, exchange rate policy, and maintain free flow of capital through its borders to ensure stable growth.” That is simply wrong, because that violates what we call the impossible trinity. What is the impossible trinity? Simply put, the impossible trinity are three choices that a government cannot choose all at the same time. It can only pick two out of these three: fixed exchange rate, free capital movement, and the ability to set its own interest rate. For example, Singapore has chosen to have free capital movement and fixed exchange rate. It must necessarily give up the ability to set its own interest rate, hence monetary policy. The States chose free capital movement and the ability to set its own interest rate, hence its exchange rate must be free floating. Why does this happen? Well, let’s consider what happens when a country chooses all 3 conditions. We prove by contradiction. Let’s imagine there’s a country called Disneyland. Suppose that Disneyland tried to do all 3. It maintains free capital movement throughits borders, it maintains a fixed exchange rate and it maintains the ability to set its own interest rate. Now let’s explore what happens when it tries to lower its interest rate to revive an economy in recession. In order to do that, Disneyland must increase the supply of money in its economy. So it increases the amount of Disney dollars in circulation, and that decreases the interest rate. However, when the interest rate is lower than that of its trading partners, all the Mickey Mice in Disneyland will want to put their money overseas, because they earn more money there with the higher interest rate there. So they sell Disney dollars in the forex market in an attempt to buy foreign currency. However, this increases the supply of Disney dollars on the forex market, and puts a downward pressure on the Disney dollar. The Disneyland government wishes to maintain a fixed exchange rate, hence it will sell foreign currency from its foreign reserves to buy back the Disney dollars that are in circulation, in effect increasing the demand for Disney dollars. However, if Disneyland keeps doing this, then it will eventually run out of foreign reserves! This is not sustainable. Also, Disneyland is merely buying back its own Disney dollars that it increased in the first place. This means that it has already lost autonomy over its own interest rate. Hence, Disneyland is forced to choose 2 out of the 3 statements. This is how the impossible trinity works in theory. In practice, people are still trying to find evidence that it exists (although it is rather robust in theory). However, it is difficult to find countries that do violate the impossible trinity and manages to do all 3 outright. Hence, it is best to keep this in mind and not to postulate that a country can do all these 3 things.
\subsection{Controlling Money Supply}
So how exactly does the government control the money supply in the market? So far, we have simply thought of the money supply as a number that the government controls. But the government cannot simply go into your wallet and take cash out from you! So how do they do it? Let’s use the United States as an example then. In the States, the central bank is called the Federal Reserve. It is actually a group of 12 central banks with many many member banks that operate together to form the central bank of the States. But for reasons of simplcity, we can simply think of the Fed as one single central bank. So how does it manipulate the money supply? There are three ways it can do so. First, by open market operations. Let’s say I’m the central bank, and you represent the US population. I sign a piece of paper called a bond, and that bond simply allows me to borrow money from you, with a promise of paying it back after a certain period with interest. I sell you that piece of paper for \$100, in effect borrowing \$100 from you. You hand me cash, and I hand you that piece of paper. I am reducing the money supply in the market because you just handed me money, while I handed you a piece of bond. Hence, this reduces the money supply in the market. Similarly, when I buy back that piece of paper for \$100, I am giving you cash and you give me back the piece of paper. This increases money supply. Through open market operations, or the sales of these bonds, the government controls the money supply. The second way is through the federal reserve ratio. All savings banks are required to keep a certain fraction of money as reserve. Let’s say you and I both operate banks. Let’s say you have \$100 in your bank. Yes don’t complain that you’re poor. \$100 bucks is all you have. Let’s say that the reserve ratio is 20\%. Then in that case, you can only loan out \$80 of your \$100 bucks to me. Now that I have \$80 bucks, I can only loan out \$64 bucks, which is 80\% out of \$80. So on and so forth. So you can see how if I increase the reserve ratio, I will reduce the amount of money circuating in the economy because more money will be locked away in the banks! The third way it manipulates money supply is through well, helicopter money. The government really literally throws money into the banks, or prints money. It doesn’t actually go to the printing press and prints more money on sheets of paper, but it does give money to the banks, sometimes by buying some of the bank’s assets like the Fed did during certain rounds of its Quantitative Easing plans. This is called helicopter money, a term given by economist Milton Friedman to describe the government throwing money out of a helicopter. 
\end{document}